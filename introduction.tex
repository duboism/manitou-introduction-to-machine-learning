\section{Introduction}

\subsection{History}

\begin{frame}
  \frametitle{A quick history}

\end{frame}

\subsection{How to go further?}

\begin{frame}
  \frametitle{General books}

  \nocite{*}

  \begin{block}{AI}<1->
    \printbibliography[heading=none,category=AI]
  \end{block}

  \begin{block}{\ac{ML}}<2->
    \printbibliography[heading=none,category=ML]
  \end{block}
\end{frame}

\begin{frame}
  \frametitle{Deep learning books}

  \nocite{*}

  \printbibliography[heading=none,category=deep_learning]
\end{frame}

\begin{frame}
  \frametitle{Courses \& tutorials}

  \begin{block}{Courses}<1->
    \begin{itemize}
    \item Coursera MOOC:
      \url{https://www.coursera.org/specializations/machine-learning-introduction} (more or less based on \href{https://cs230.stanford.edu/}{Stanford CS230 (Deep Learning)})
    \item FIDLE (Formation Introduction au Deep LEarning): \url{https://fidle.cnrs.fr/}
      (yearly course with practical examples in French)
    \end{itemize}
  \end{block}

  \begin{block}{Challenges}<2->
    \begin{itemize}
    \item Kaggle: \url{https://www.kaggle.com/}
    \item ENS Challenge Data: \url{https://challengedata.ens.fr/}
    \end{itemize}
  \end{block}

  \begin{block}{Tutorials}<3->
    \begin{itemize}
    \item PyTorch tutorials: \url{https://pytorch.org/tutorials/} (in-depth
      tutorials about PyTorch)
    \item scikit-lear tutorials:
      \url{https://scikit-learn.org/stable/user_guide.html} (in depth tutorials
      about scikit-learn)
    \end{itemize}
  \end{block}
\end{frame}
