\section{General introduction}

\subsection{Problems, tasks and approaches}

\begin{frame}

  \frametitle{General setting}

  \begin{textblock}{90}(5,15)
    % Inspired by figure 3.4 in \cite{barra2021} (p. 82)
    \begin{tikzpicture}
      \tikzstyle{line} = [draw, -latex', thick]
      \def\examples{$\example_1, \example_2, \cdots, \example_{\nTrainingSamples}$}
      \def\labels{$\knownLabel_1, \knownLabel_2, \cdots, \knownLabel_{\nTrainingSamples}$}
      \def\predictions{${\footnotesize\apply{\hyp_{\params}}{\example_1}, \cdots, \apply{\hyp_{\params}}{\example_{\nTrainingSamples}}}$}

      % Environment
      \def\env{Environment}
      \node[minimum width=2.5cm,ellipse,draw,thick] (\env) at (0, 0) {\env{} $\exampleSet$};

      % Oracle
      \onslide<1-3>{
        % Node
        \def\oracle{Oracle}
        \node[ellipse,draw,thick,right = 2.5cm of \env] (\oracle) {\oracle{}};
      }

      % Learner
      \def\learner{Learner}
      \node[minimum width=2.5cm,draw,thick,below = 1.5cm of \oracle] (\learner) {\learner{} $\hyp_{\params} \in \hypSet$};

      % Output of learner
      \def\output{Output}
      \node[right = 0.3cm of \learner] (\output) {\predictions};

      % \env -> \oracle & \env -> \learner
      \onslide<1-3>{
        \path[line] (\env) -- node[above=0.2cm] {\examples} (\oracle);
      }
      \path[line] (\env) -- +(0,-1) |- node[below=0.2cm] {\examples} (\learner);

      % \oracle -> learner
      \onslide<1-3>{
        \path[line] (\oracle) -- node[right=0.1cm] {\labels} (\learner);
      }

      % \learner -> \output
      \draw[line] (\learner) -- (\output);
    \end{tikzpicture}
  \end{textblock}

  \begin{textblock}{90}(5,55)
    \only<1>{
    Ingredients:
    \begin{itemize}
    \item The environment generates $\nTrainingSamples$ examples
      $\drawFrom{\example}{\pdfExample}$ (unknown)
    \item The oracle labels them according to a function $\targetFunc: \exampleSet
      \mapsto \labelSet$ or a probability distribution $\condpdfLabelExample$ (unknown)
    \item The learner has a set of functions $\exampleSet
      \mapsto \labelSet$, noted $\hypSet$
    \item The goal is to find $\opt{\hyp} \in \hypSet$ that approximates
      $\targetFunc$
      \begin{itemize}
      \item Note that it's possible that $\targetFunc  \not\in \hypSet$
      \end{itemize}
    \end{itemize}
    }

    \only<2>{
      Different protocols:
      \begin{itemize}
      \item Batch learning: all $\nTrainingSamples$ examples are given
      \item Online learning: examples are given one by one
      \item More complex:
        \begin{itemize}
        \item Active learning: the learner searches for examples
        \item Reinforcement learning
        \end{itemize}
      \end{itemize}
    }
    \only<3->{
      Different tasks:
      \begin{columns}[T]
        \begin{column}{0.5\textwidth}
          \begin{itemize}
          \item Supervised learning:
            \begin{itemize}
            \item Classification: $\labelSet = \setext{1,..,\nClasses}$
            \item Multi-classification
            \item Regression: $\labelSet \subset \setR^n$
            \item Ranking
            \end{itemize}
          \item<4-> Unsupervised learning:
            \begin{itemize}
            \item Clustering: find groups in $\exampleSet$
            \item Density estimation: find $\pdfExample$
            \end{itemize}
          \end{itemize}
        \end{column}
        \begin{column}{0.5\textwidth}
          \begin{itemize}
          \item<5-> More complex tasks:
            \begin{itemize}
            \item Self-supervised
            \item Semi-supervised
            \end{itemize}
          \end{itemize}
        \end{column}
      \end{columns}
    }
  \end{textblock}
\end{frame}


\begin{frame}
  \frametitle{How to approximate $\targetFunc$? (1/2)}

  \begin{textblock}{90}(5, 10)
    \begin{block}{True risk}
      Suppose that we have a loss function $\lossFunc: \exampleSet \times \labelSet \mapsto
      \setR^+$ that evaluates how bad one example is classified by an hypothesis $\hyp$.
      The true risk of $\hyp$ is then:
      \begin{equation}
        \begin{split}
          \apply{\risk}{\hyp} & = \expectation{\apply{\lossFunc}{\apply{\hyp}{\example}, \predLabel}}\\
                              & = \int_{\example \in \exampleSet, \predLabel \in \exampleSet}
          \apply{\lossFunc}{\apply{\hyp}{\example}, \predLabel} \pdfExampleLabel \intover{\example} \intover{\predLabel}
        \end{split}
      \end{equation}
    \end{block}
\end{textblock}

  \begin{textblock}{90}(5, 45)
    \begin{block}{Examples of loss functions}
      \begin{itemize}
      \item For classification:
        \begin{itemize}
        \item 0-1 loss:
          $\apply{\lossFunc}{\apply{\hyp}{\example}, \predLabel} =
          \begin{cases}
            0 & \text{if $\apply{\hyp}{\example} \ne \predLabel$} \\
            1 & \text{if $\apply{\hyp}{\example}  =  \predLabel$}
          \end{cases}
          $
        \item Logistic loss (binary probabilistic classification): $\apply{\lossFunc}{\apply{\hyp}{\example},
            \predLabel} = -\apply{\log}{\frac{\apply{\hyp}{\example}}{1-\apply{\hyp}{\example}}}$
        \end{itemize}
      \item For regression:
        \begin{itemize}
        \item Quadratic: $\apply{\lossFunc}{\apply{\hyp}{\example}, \predLabel}
          = \left( \apply{\hyp}{\example} - \predLabel \right)^2$
        \end{itemize}
      \end{itemize}
    \end{block}
  \end{textblock}
\end{frame}


\begin{frame}
  \frametitle{How to approximate $\targetFunc$? (2/2)}

  \begin{textblock}{90}(5, 10)
    \begin{block}{Optimal hypothesis}
      The best hypothesis $\opt{\hyp}$ is:
      \[
        \opt{\hyp} = \argmin_{\hyp \in \hypSet} \apply{\risk}{\hyp}
      \]
    \end{block}
  \end{textblock}

  \begin{textblock}{90}(5, 35)
    \begin{block}{Generative \vs{} Discriminative approach}
      \begin{itemize}
      \item Generative learning
      \item Discriminative learning
      \end{itemize}
    \end{block}
  \end{textblock}
\end{frame}


\subsection{Theoretical approaches}

\begin{frame}
  \frametitle{Theoretical approaches}

  \begin{textblock}{100}(5,10)
    \begin{block}{Test}
      \begin{itemize}
      \item Several approaches
      \item Mainly developed for supervised learning
      \end{itemize}
    \end{block}
  \end{textblock}
\end{frame}
