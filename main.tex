\documentclass[10pt]{beamer}

% Packages
\usepackage[utf8]{inputenc}
\usepackage[OT1]{fontenc}
\usepackage[english]{babel}
\usepackage{pgfpages}
\usepackage[style=authortitle,giveninits=true,sorting=none,backend=biber]{biblatex}
\usepackage[absolute,overlay]{textpos}
\usepackage{ifdraft}
\usepackage{tikz}
\usetikzlibrary{positioning}
\usetikzlibrary{matrix}
\usetikzlibrary{arrows.meta}
\usepackage{pgfplots}
% Trick to make pgfplots work with aux_dir
% (https://tex.stackexchange.com/a/444407)
% This uses the value in latexmkrc (not great, not terrible)
\pgfkeys{/pgf/plot/gnuplot call={cd build && gnuplot}}
\usepackage{mathtools} % for aligned
\usepackage{listofitems} % for \readlist to create arrays
\usepackage{stmaryrd}
\usepackage{acronym}
\usepackage{algorithm2e}

%
% General commands
%
% $Author: duboism $
% $LastChangedDate: 2011-06-21 21:26:47 +0200 (mar. 21 juin 2011) $

% TODO: better way to do that:
%  - find the current language (babel var?) and then apply a "foreign word" style
% (would work whatever the current language is)

% Latin words
\def\latinstyle{\textit}
\newcommand{\latin}[1]{\latinstyle{#1}}
\newcommand{\ie}{\latin{i.e.}}
\newcommand{\eg}{\latin{e.g.}}
\newcommand{\cf}{\latin{cf.}}
\newcommand{\etc}{\latin{etc.}}
\newcommand{\vs}{\latin{versus}}

\def\greekstyle{\textit}
\newcommand{\greek}[1]{\greekstyle{#1}}

% english words
\let\engstyle\textit
\newcommand{\english}[1]{\foreignlanguage{english}{\engstyle{#1}}}

% table titles
% TODO: better ways of doing that:
% - as an option of tabular?
% - only one command (whatever the mode)
\let\tabletitlestyle\textbf
\let\mathtabletitlestyle\mathbf
\newcommand{\tabletitle}[1]{\tabletitlestyle{#1}}
\newcommand{\mathtabletitle}[1]{\ensuremath{\mathtabletitlestyle{#1}}}

% Index
\def\indexstyle{\textbf}
\newcommand{\putindex}[1]{\indexstyle{#1}\index{#1@#1}}
\newcommand{\putindexsee}[2]{\indexstyle{#1}\index{#1@#1|see{#2}}}
\newcommand{\putindexcat}[2]{\indexstyle{#1}\index{#2@#2!{#1}}}

%
% Math commands
%
% $Author: duboism $
% $LastChangedDate: 2013-01-15 16:56:30 +0100 (mar. 15 janv. 2013) $

%\usepackage{amsmath, amssymb, amsthm, textcomp, stmaryrd}
%\usepackage{textcase}

%%%%%%%%%%%
% General %
%%%%%%%%%%%

%\newtheorem{definition}{Définition}
%\newtheorem{thm}{Théorème}
%\newtheorem{prop}{Proposition}
%\newtheorem{exemple}{Exemple}
%\newtheorem{NB}{NB}

% Sets
\def\setextleftdelim{\lbrace}
\def\setextrightdelim{\rbrace}
\newcommand{\setext}[1]{\ensuremath{\left \setextleftdelim {#1} \right \setextrightdelim}} % Ensemble en extension: {x1, ..., xn}
\def\setpredicatesep{,}
\newcommand{\setpredicate}[2]{\ensuremath{\left \setextleftdelim {#1} \setpredicatesep {#2} \right \setextrightdelim}} % Ensemble en intension: {xi, 1<= i <= n}
\def\setstyle{\mathcal}
\newcommand{\set}[1]{\ensuremath{\setstyle{#1}}}
\newcommand{\card}[1]{\ensuremath{\left | #1 \right |}}

% Well known sets
\def\setN{\mathbb{N}}
\def\setZ{\mathbb{Z}}
\def\setQ{\mathbb{Q}}
\def\setR{\mathbb{R}}
\def\setC{\mathbb{C}}

% Tuples
\def\tupleleftdelim{(}
\def\tuplerightdelim{)}
\newcommand{\couple}[2]{\ensuremath{\left \tupleleftdelim {#1}, {#2} \right \tuplerightdelim}}
\newcommand{\tupleext}[1]{\ensuremath{\left \tupleleftdelim {#1} \right \tuplerightdelim}} % Tuple en extension: (x1, ..., xn)
\def\tupleintsep{_}
\newcommand{\tupleint}[2]{\ensuremath{\left \tupleleftdelim {#1} \right \tuplerightdelim \tupleintsep {#2}}} % Tuple en intension: (xi)_{1 <= i <= n}
\def\tuplestyle{\mathbf}
\newcommand{\tuple}[1]{\ensuremath{\tuplestyle{#1}}}


% Intervals: http://en.wikipedia.org/wiki/Interval_%28mathematics%29#Notations_for_intervals
\def\intervalleftdelim{[}
\def\intervalrightdelim{]}
\def\intervallodelim{\intervalrightdelim} % left-open symbol
\def\intervalrodelim{\intervalleftdelim} % right-open symbol
\newcommand{\interval}[2]{\ensuremath{\left \intervalleftdelim {#1}, {#2} \right \intervalrightdelim}}
\newcommand{\lointerval}[2]{\ensuremath{\left \intervallodelim {#1}, {#2} \right \intervalrightdelim}}
\newcommand{\rointerval}[2]{\ensuremath{\left \intervalleftdelim {#1}, {#2} \right \intervalrodelim}}
\newcommand{\ointerval}[2]{\ensuremath{\left \intervallodelim {#1}, {#2} \right \intervalrodelim}}

% Integer intervals
\def\intintervalleftdelim{\llbracket}
\def\intintervalrightdelim{\rrbracket}
\def\intintervallodelim{\intintervalrightdelim} % left-open symbol
\def\intintervalrodelim{\intintervalleftdelim} % right-open symbol
\newcommand{\intinterval}[2]{\ensuremath{\left \intintervalleftdelim {#1}, {#2} \right \intintervalrightdelim}}
\newcommand{\lointinterval}[2]{\ensuremath{\left \intintervallodelim {#1}, {#2} \right \intintervalrightdelim}}
\newcommand{\rointinterval}[2]{\ensuremath{\left \intintervalleftdelim {#1}, {#2} \right \intintervalrodelim}}
\newcommand{\ointinterval}[2]{\ensuremath{\left \intintervallodelim {#1}, {#2} \right \intintervalrodelim}}

% Functions and operators
\def\functionstyle{\mathrm}
\newcommand{\func}[1]{\ensuremath{\functionstyle{#1}}}
\newcommand{\apply}[2]{\ensuremath{#1 \left ( {#2} \right )}}
\def\opstyle{\mathrm}
\newcommand{\op}[1]{\ensuremath{\opstyle{#1}}}
\newcommand{\applyop}[2]{\ensuremath{#1 \left [ {#2} \right ]}}

% Vectors, Matrices, ...
\def\vectorextleftdelim{[}
\def\vectorextrightdelim{]}
\newcommand{\vectorext}[1]{\ensuremath{\left \vectorextleftdelim {#1} \right \vectorextrightdelim}}
\newcommand{\vectorstyle}[1]{\mathbf{#1}}
\renewcommand{\vector}[1]{\ensuremath{\vectorstyle{#1}}}
\newcommand{\vectorElem}[2]{\ensuremath{#1 \left [ #2 \right ]}}

\def\matrixextleftdelim{[}
\def\matrixextrightdelim{]}
\newcommand{\matrixext}[1]{\ensuremath{\left \matrixextleftdelim {#1} \right \matrixrightdelim}}
\def\matrixstyle{\mathbf}
\renewcommand{\matrix}[1]{\ensuremath{\matrixstyle{#1}}}
\newcommand{\matrixElem}[3]{\ensuremath{#1 \left [ {#2}, {#3} \right ]}}
\newcommand{\matrixLinElem}[2]{\ensuremath{#1 \left [ #2 \right ]}}

% Scalar product
\def\dotprodleftdelim{\left \langle}
\def\dotprodrightdelim{\right \rangle}
\def\dotprodcentraldelim{,}
\newcommand{\dotprod}[2]{\ensuremath{\dotprodleftdelim #1 \dotprodcentraldelim #2 \dotprodrightdelim}}

% Vector product
\def\vectorprodsymbol{\times}
\newcommand{\vectorprod}[2]{#1 \vectorprodsymbol #2}

% TODO:
% - sets: set builder notation? http://en.wikipedia.org/wiki/Set_builder_notation
% - set theory: define synonyms for common opertaions (union <-> \bigcup, intersection <-> \bigcap, etc.)
% - matrix
% - indexing should be done on class basis: setelement, matrix element, imageelement, ...
% - limits: use \def instead of \newcommand?
% - probabilities: use \DeclareMathOperator? There is a \Pr operator in \LaTeX

% Integral
\def\dsymb{\mathrm{d}}
\newcommand{\intover}[1]{\dsymb #1}

% Limits
\newcommand{\tendsto}[1]{\ensuremath{\rightarrow #1}}
\newcommand{\tendstoP}{\ensuremath{\overset{P}{\rightarrow}}}

% Optimal quantity
\def\optsymbol{\star}
\newcommand{\opt}[1]{\ensuremath{\overset{\optsymbol}{#1}}}

% Estimation of a quantity
\def\estimsymbol{\widehat}
\newcommand{\estim}[1]{\ensuremath{\estimsymbol{#1}}}

%%%%%%%%%%%%%%%%%
% Probabilities %
%%%%%%%%%%%%%%%%%
\newcommand{\RV}[1]{\ensuremath{\MakeTextUppercase{#1}}}
\newcommand{\cond}[2]{{#1} | {#2}}


% Discrete probability
\def\probSymbol{P}
\newcommand{\prob}[1]{\ensuremath{\apply{\func{\probSymbol}}{#1}}}
\newcommand{\condprob}[2]{\ensuremath{\prob{\cond{#1}{#2}}}}

% PDF
\def\pdfSymbol{p}
\newcommand{\pdf}[1]{\ensuremath{\apply{\func{\pdfSymbol}}{#1}}}
\newcommand{\condpdf}[2]{\ensuremath{\pdf{\cond{#1}{#2}}}}

% Operators
\newcommand{\expectation}[1]{\ensuremath{\applyop{\op{\mathbb{E}}}{#1}}}
\newcommand{\var}[1]{\ensuremath{\applyop{\op{Var}}{#1}}}


%%%%%%%%%%%%%%%%%%%%%%
% General operations %
%%%%%%%%%%%%%%%%%%%%%%

% Argmax & friends
\DeclareMathOperator*{\argmax}{arg\,max}
\DeclareMathOperator*{\argmin}{arg\,min}

% BigO and friends
% From: http://www.tug.org/pipermail/texhax/2004-January/001496.html
% See also: http://xw2k.nist.gov/dads//HTML/bigOnotation.html
\DeclareMathOperator{\BigOmicron}{O}
\DeclareMathOperator{\littleomicron}{o}
\DeclareMathOperator{\BigOmega}{\Omega}
\DeclareMathOperator{\littleomega}{\omega}
\DeclareMathOperator{\BigTheta}{\Theta}

\newcommand{\boundedabove}[1]{\BigOmicron \left ( {#1} \right )}
\newcommand{\dominated}[1]{\littleomicron \left ( {#1} \right )}

\newcommand{\boundedbelow}[1]{\BigOmega \left ( {#1} \right )}
\newcommand{\dominates}[1]{\littleomega \left ( {#1} \right )}

\newcommand{\boundedabovebelow}[1]{\BigTheta \left ( {#1} \right )}
\newcommand{\similar}[2]{{#1} \sim {#2}}

% Norms
\def\normleftdelim{\|}
\def\normrightdelim{\|}
\newcommand{\pnorm}[2]{\ensuremath{\left \normleftdelim {#1} \right \normrightdelim_{#2}}}
\newcommand{\norm}[1]{\ensuremath{\left \normleftdelim {#1} \right \normrightdelim}}
\newcommand{\abs}[1]{\ensuremath{\left | {#1} \right |}}

% Special functions
\newcommand{\ceil}[1]{\ensuremath{\left \lceil {#1} \right \rceil}}
\newcommand{\floor}[1]{\ensuremath{\left \lfloor {#1} \right \rfloor}}
\DeclareMathOperator{\sgn}{sgn}

% Convolution
\def\convsymbol{\star}
\DeclareMathOperator*{\conv}{\convsymbol}
\newcommand{\convvar}[1]{\ensuremath{\conv_{#1}}} % argument is used to specify convolution variables

%%%%%%%%%%%%%%%%%%%%
% Image processing %
%%%%%%%%%%%%%%%%%%%%

\let\imagestyle\mathbf
\newcommand{\image}[1]{\ensuremath{\imagestyle{#1}}}
\newcommand{\imagepixel}[3]{\ensuremath{\image{#1} \left ( #2, #3 \right )}}
\newcommand{\gausskernel}[3]{\ensuremath{\frac{1}{\sqrt{2\pi} #3} \exp^{-\frac{(#1+#2)^2}{2 #3^2}} }}

% SIFT
\newcommand{\siftkernel}{\gausskernel{x}{y}{\sqrt{2}}}

%\newcommand{\convdd}[4]{\int_3 \int_4 #1 #2}

%%%%%%%%%%%%
% Learning %
%%%%%%%%%%%%

\DeclareMathOperator{\error}{error}

% Probabilistic robotics
\newcommand{\stateSymbol}{x}
\newcommand{\state}{\vector{\stateSymbol}}
\newcommand{\obsSymbol}{z}
\newcommand{\obs}{\vector{\obsSymbol}}
\newcommand{\controlSymbol}{u}
\newcommand{\control}{\vector{\controlSymbol}}
\newcommand{\aggState}{\tilde{x}}
\newcommand{\stateRVSymbol}{\RV{\stateSymbol}}
\newcommand{\stateRV}{\vector{\stateRVSymbol}}
\newcommand{\obsRVSymbol}{\RV{\obsSymbol}}
\newcommand{\obsRV}{\vector{\obsRVSymbol}}
\newcommand{\controlRVSymbol}{\RV{\controlSymbol}}
\newcommand{\controlRV}{\vector{\controlRVSymbol}}
\newcommand{\aggStateRV}{\RV{\aggState}}

\newcommand{\val}[2]{\ensuremath{{#1}_{#2}}}
\newcommand{\seq}[3]{\ensuremath{{#1}_{{#2}:{#3}}}}

\newcommand{\bel}{\func{bel}}
\newcommand{\belBeforeObs}{\overline{{\bel}}}



% Options, definitions and styles
\setbeameroption{show notes on second screen}
\DeclareFontShape{T1}{cmss}{b}{n}{<->ssub * cmss/bx/n}{}
\AtBeginSection[]{%
  \begin{frame}
    \frametitle{Outline}
    \tableofcontents[currentsection,hideothersubsections]
  \end{frame}
}
\AtBeginSubsection[]{%
  \begin{frame}
    \frametitle{Outline}
    \tableofcontents[currentsection,subsectionstyle=show/shaded/hide]
  \end{frame}
}
% Grid
\TPGrid{100}{100}
\ifdraft
  {%
    \TPoptions{%
      showboxes=true,%
    }%
  }%
  {%
  }
% Hyperref
\hypersetup{
  colorlinks=true,
  linkcolor=blue,
  filecolor=magenta,
  urlcolor=cyan,
}
% TiKz
\tikzset{%
  every neuron/.style={
    circle,
    draw,
    %minimum size=1cm
  },
  neuron missing/.style={
    draw=none,
    scale=2.5,
    text height=0.333cm,
    execute at begin node=\color{black}$\vdots$
  },
}
% Bibliography and citation style
\addbibresource{references.bib}
\DeclareNameAlias{author}{last-first}
\DeclareBibliographyCategory{AI}
\addtocategory{AI}{stuart2021}
\DeclareBibliographyCategory{ML}
\addtocategory{ML}{bishop2006}
\addtocategory{ML}{duda2001}
\addtocategory{ML}{james2023}
\addtocategory{ML}{barra2021}
\DeclareBibliographyCategory{deep_learning}
\addtocategory{deep_learning}{goodfellow2016}
\addtocategory{deep_learning}{zhang2023}
\addtocategory{deep_learning}{chollet2021}

% Notations
\def\trainingSet{\set{S}}
\def\exampleSet{\set{X}}
\def\nTrainingSamples{m}
\def\example{\vector{x}}
\def\dimExample{n}
\def\labelSet{\set{Y}}
\def\knownLabel{u}
\def\predLabel{y}
\def\nClasses{K}
\def\hyp{\func{h}}
\def\params{\vector{\theta}}
\def\output{o}
\def\layer{\ell}
\def\weight{w}
\def\weightMatrix{\matrix{W}}
\def\activFunc{\func{\sigma}}
\def\ReLUFunc{\func{ReLU}}
\def\ReLUFuncExt{\func{\max}}
\def\softMaxFunc{\func{SoftMax}}
\def\lossFunc{\func{L}}
\def\learningRate{\alpha}
\DeclareMathOperator{\grad}{\nabla}

% Acronyms
\acrodef{AI}{Artificial Intelligence}
\acrodef{xAI}{eXplainable AI}
\acrodef{ML}{Machine Learning}
\acrodef{DL}{Deep Learning}
\acrodef{MLP}{Multi-Layer Perceptron}
\acrodef{CNN}{Convolutional Neural Networks}

\title{Introduction to Machine Learning}
\author{Sylvain Caillou \and Mathieu Dubois}
\institute{L2IT}
\date{2024-07-05}

\begin{document}

\frame{\titlepage}


\begin{frame}
  \note{
    \begin{itemize}
    \item Limits and approach of our talk
    \item Introduce Natalia's talk
    \end{itemize}
  }
  \frametitle{Goal}

  \begin{textblock}{90}(5, 15)
    \begin{itemize}
    \item \aclu{ML} is a vast and expanding universe \onslide<2->{with $\Lambda
        > 0$}
    \item<3-> We can't talk about everything:
      \begin{itemize}
      \item<4-> We will start with simple examples and then generalize
      \item<4-> We won't talk much about GW
      %\item<5-> At the end you should be able to understand \acs{ML} articles
      \end{itemize}
    \item<5-> Next talk will introduce more advanced models in the context of GW
    \end{itemize}
  \end{textblock}
\end{frame}


\begin{frame}
  \frametitle{Outline}
  \tableofcontents[hidesubsections]
\end{frame}

\section{Introduction}

\subsection{History}

\begin{frame}
  \frametitle{A quick history}

\end{frame}

\subsection{How to go further?}

\begin{frame}
  \frametitle{General books}

  \nocite{*}

  \begin{block}{AI}<1->
    \printbibliography[heading=none,category=AI]
  \end{block}

  \begin{block}{\ac{ML}}<2->
    \printbibliography[heading=none,category=ML]
  \end{block}
\end{frame}

\begin{frame}
  \frametitle{Deep learning books}

  \nocite{*}

  \printbibliography[heading=none,category=deep_learning]
\end{frame}

\begin{frame}
  \frametitle{Courses \& tutorials}

  \begin{block}{Courses}<1->
    \begin{itemize}
    \item Coursera MOOC:
      \url{https://www.coursera.org/specializations/machine-learning-introduction} (more or less based on \href{https://cs230.stanford.edu/}{Stanford CS230 (Deep Learning)})
    \item FIDLE (Formation Introduction au Deep LEarning): \url{https://fidle.cnrs.fr/}
      (yearly course with practical examples in French)
    \end{itemize}
  \end{block}

  \begin{block}{Challenges}<2->
    \begin{itemize}
    \item Kaggle: \url{https://www.kaggle.com/}
    \item ENS Challenge Data: \url{https://challengedata.ens.fr/}
    \end{itemize}
  \end{block}

  \begin{block}{Tutorials}<3->
    \begin{itemize}
    \item PyTorch tutorials: \url{https://pytorch.org/tutorials/} (in-depth
      tutorials about PyTorch)
    \item scikit-lear tutorials:
      \url{https://scikit-learn.org/stable/user_guide.html} (in depth tutorials
      about scikit-learn)
    \end{itemize}
  \end{block}
\end{frame}


\section{A simple example: classify handwritten digits with a multi-layer
  perceptron}

\begin{frame}
  \frametitle{The MNIST dataset}
  \begin{columns}
    \begin{column}{0.5\textwidth}
      \begin{itemize}
      \item Classic dataset
        \begin{itemize}
        \item Lots of methods have been tested with it
        \end{itemize}
      \item All images have the same size and aspect ratio (non-trivial
        pre-processing)
      \item The label is known: $\trainingSet = \setpredicate{\couple{\example_{i}}{\knownLabel_{i}}}{i=1..\nTrainingSamples}$
      \item The task is to recognize the digit on the image \ie{} we are looking
        for a function: $\decisionFunc: \, \setR^{\dimExample} \mapsto \intinterval{0}{9}$
      \end{itemize}
    \end{column}
    \begin{column}{0.5\textwidth}
      \begin{tikzpicture}[>=stealth]
        \node[anchor=south west,inner sep=0] (image) at (0,0) {
          \includegraphics[width=\columnwidth]{img/MnistExamples.png}
        };
        \onslide<2->{
        \begin{scope}[x={(image.south east)},y={(image.north west)}]
          \draw[red,ultra thick] (0.22,0.15) rectangle (0.28,0.25);
        \end{scope}
        }
      \end{tikzpicture}
      \url{http://yann.lecun.com/exdb/mnist/index.html}
    \end{column}
  \end{columns}
\end{frame}

\begin{frame}
  \frametitle{Multi-layer perceptron: computation}
  \begin{textblock}{50}(0,10)
    \begin{itemize}
    \item 1957 but many improvements since
    \item Very loose biological inspiration
    \end{itemize}
  \end{textblock}
  \begin{textblock}{100}(5,20)
    \begin{tikzpicture}[>=stealth]
      % Inspired by https://tex.stackexchange.com/a/153974
      % We display a MLP with 2 hidden layers for OCR
      % The input image and the output layer are always displayed
      % The layers (input and the 2 hidden) and their connections appear after
      % Description of layers are on the graph

      % Parameters
      \def\xInputImage{0}
      \def\nameInputImage{input-name}
      \def\xInput{2.2}
      \def\nameInput{input}
      \def\xHiddenI{4.4}
      \def\nameHiddenI{hidden1}
      \def\xHiddenII{6.6}
      \def\nameHiddenII{hidden2}
      \def\xOutput{8.8}
      \def\nameOutput{output}
      \def\xProba{9.2}
      \def\nameProba{proba}
      \def\missing{missing}

      % Input image and grid
      % This part is inspired by https://tex.stackexchange.com/a/128648
      % and a lot of trial and error for the grid (including a bit of ChatGPT)
      % Apparently, the idea is to fix the size of the image and then the grid
      % flows
      % I use 56px (twice the real size) so that it looks OK and computations are easy.
      \node[anchor=center,inner sep=0pt,draw=black] (\nameInputImage) at (\xInputImage,0) {
        \includegraphics[width=56px,height=56px]{img/Mnist_8.png}
      };
      \begin{scope}
        \clip (\nameInputImage.south west) rectangle (\nameInputImage.north east);
        \draw[step=2px,gray,very thin] (\nameInputImage.south west) grid (56px, 56px);
      \end{scope}

      \onslide<2->{
      % Input layer
      \node [align=center, above] at (\xInput,2.2) {%Input \\ layer \\
          {\tiny $28\times28=784$ units}};
      \foreach \m/\l [count=\y] in {1,2,3,missing,784}
      \node [every neuron/.try, neuron \m/.try] (\nameInput-\m) at (\xInput,3.0-\y) {};

      % Connections from input image to input layer
      % No loop here cause we need to shift the positions
      % It uses the scale introduced above
      \draw [->] (\nameInputImage)++(-27px,27px) -- (\nameInput-1);
      \draw [->] (\nameInputImage)++(-25px,27px) -- (\nameInput-2);
      \draw [->] (\nameInputImage)++(-23px,27px) -- (\nameInput-3);
      \draw [->] (\nameInputImage)++(27px,-27px) -- (\nameInput-784);

      % Description of the output
      \node[anchor=north west] at (\xInput-1, -2.2) {$\output^{1}_{k} = \matrixLinElem{\matrix{I}}{k}$} ;
      }

      % First hidden layer
      % Connections are displayed in 2 different slides (see below)
      \onslide<3->{
      \node [align=center, above] at (\xHiddenI,2.2) {%Hidden \\ layer \\
          {\tiny 200 $\ReLUFunc$ units}};
      \foreach \m [count=\y] in {1,2,missing,3}
      \node [every neuron/.try, neuron \m/.try ] (\nameHiddenI-\m) at (\xHiddenI,2.5-\y) {};
      }

      % Connections from input layer to first neuron with weights
      % and explanation of the output
      \onslide<3>{
      \foreach \p in {1,2,3,784}
      \foreach \k in {1}
      \draw[->] (\nameInput-\p) -- node {${\color{red} \weight^{2}_{\k,\p}}$} (\nameHiddenI-\k);

      % Description of the output
      \node[text width=5,anchor=north west] at (\xHiddenI-1, -2.2) {{\footnotesize
          \begin{align*}
            \begin{aligned}
              & s^{2}_{k}      & = & \sum_{p=1}^{N_{1}} {\color{red} \weight^{2}_{k,p}} \output^{1}_{p} \\
              & \output^{2}_{k} & = & \ \apply{\activFunc}{s^{2}_{k}} \\
              &                     & = & \apply{\ReLUFuncExt}{0, s^{2}_{k}}
            \end{aligned}
          \end{align*}
        }};
      }

      % Other connections from input layer to first hidden layer
      % and matrix form
      \onslide<4->{
      \foreach \i in {1,2,3,784}
      \foreach \j in {1,...,3}
      \draw[->] (\nameInput-\i) -- (\nameHiddenI-\j);
      \node[text width=3] at (\xHiddenI-1, -2.4) {{
          \begin{equation*}
            %\begin{aligned}
            \vector{\output^{2}} = \apply{\activFunc}{{\color{red}\weightMatrix^{2}}\vector{\output^{1}}}
            %\end{aligned}
          \end{equation*}
        }};
      }

      % Second hidden layer
      \onslide<5->{
      \node [align=center, above] at (\xHiddenII,2.2) {% Hidden \\ layer \\
          {\tiny 150 $\ReLUFunc$ units}};
      \foreach \m [count=\y] in {1,2,missing,3}
      \node [every neuron/.try, neuron \m/.try ] (\nameHiddenII-\m) at (\xHiddenII,2.2-0.9*\y) {};

      % Connections to first hidden layer to second hidden layer
      \foreach \i in {1,2,3}
      \foreach \j in {1,2,3}
      \draw [->] (\nameHiddenI-\i) -- (\nameHiddenII-\j);

      \node[text width=3,anchor=north west] at (\xHiddenII-1, -2.2) {{
          \begin{equation*}
            %\begin{aligned}
            \vector{\output^{3}} = \apply{\activFunc}{{\color{red}\weightMatrix^{3}}\vector{\output^{2}}}
            %\end{aligned}
          \end{equation*}
        }};
      }

      % Output layer
      \node [align=center, above] at (\xOutput,2.2) {% Output \\ layer \\
        {\tiny 10 $\softMaxFunc$ units}};
      \foreach \m [count=\y] in {0,1,missing,9}
      \node [every neuron/.try, neuron \m/.try ] (\nameOutput-\m) at (\xOutput,1.8-0.7*\y) {};

      \onslide<6->{
      % Connection from second hidden layer to output layer
      \foreach \i in {1,2,3}
      \foreach \j in {0,1,9}
      \draw [->] (\nameHiddenII-\i) -- (\nameOutput-\j);

      % Description of output
      \node[text width=3,anchor=north west] at (\xOutput-1, -2.2) {{\footnotesize
          \begin{align*}
            \begin{aligned}
              & s^{\layer+1}_{k}      & = & \sum_{p=1}^{N_{\layer}} {\color{red} \weight^{\layer}_{k,p}} \output^{\layer}_{p} \\
              & \output^{n}_{k} & = & \frac{\apply{\exp}{s^{n}_{k}}}{\sum_{j=1}^{10} \apply{\exp}{s^{n}_{j}}}
            \end{aligned}
          \end{align*}
        }};
      }

      % Output probabilities
      \foreach \m [count=\y] in {0,1,missing,9}
      \node (\nameProba-\m) at (\xProba,1.8-0.7*\y) {\ifx\m\missing\else {\tiny $\prob{\m}$}\fi};
    \end{tikzpicture}

  \end{textblock}

\end{frame}




\section{A more convolved example: \acl{CNN}}


\begin{frame}
  \frametitle{\acl{CNN}: Introduction}

  \begin{textblock}{90}(5, 15)
    \begin{itemize}
    \item Introduced in 1989 - 1995
    \item Specialized for image-like structures
    \item Once again roughly biologically inspired
    \item Influential in the rise of deep-learning
    \item Same task than in the previous section
    \end{itemize}
  \end{textblock}
\end{frame}


\begin{frame}[label=CNN_Overview]
  \note{
    \begin{itemize}
    \item Feed-forward
    \item Special layers
    \end{itemize}
  }
  \frametitle{\acl{CNN}: Computation}

  \begin{textblock}{90}(5,10)
    \begin{center}
      \includegraphics[width=\textwidth]{img/CNN.png}
      Taken from FIDLE
    \end{center}
  \end{textblock}

  \begin{textblock}{45}(5,60)
    \begin{itemize}
    \item Feed-forward architecture
    \item Special layers
      \begin{itemize}
      \item<2-> \hyperlink{Convolutional_Layers}{Convolutional layers}
        \onslide<3->{
          \begin{itemize}
          \item Kernels are learned
          \end{itemize}
        }
      \item<4-> \hyperlink{Pooling_Layers}{Pooling layers}: no weights
      \end{itemize}
    \item<5-> \ac{MLP}
    \end{itemize}
  \end{textblock}

  \begin{textblock}{45}(50,60)
    \onslide<6->{
      \begin{itemize}
      \item The idea is that the layers extract more and more abstract features
      \item Weights are shared: less parameters to learn
      \item Use the \emph{structure} of the image
      \end{itemize}
    }
  \end{textblock}
\end{frame}


% TODO: hyperlinks
\begin{frame}[label=Convolutional_Layers]
  \frametitle{\acl{CNN}: Convolutional Layers}

  \begin{textblock}{90}(5, 10)
    \begin{center}
      \includegraphics[width=\textwidth]{img/Convolution.png}
      Taken from FIDLE
    \end{center}
  \end{textblock}

  % Cheat a bit on the size
  \begin{textblock}{46}(5, 75)
    \begin{itemize}
    \item Common operation in image processing
    \item $\matrix{K}$ acts as a filter
    \end{itemize}
    %\hyperlink{CNN_Overview}{\beamergotobutton{Go back}}
  \end{textblock}

  \begin{textblock}{45}(50, 75)
    \includegraphics[width=\textwidth]{img/CNN_Filter.png}
  \end{textblock}
\end{frame}


% TODO: hyperlinks
\begin{frame}[label=Pooling_Layers]
  \frametitle{\acl{CNN}: Pooling Layers}

  \begin{textblock}{90}(5, 10)
    \begin{center}
      \includegraphics[width=\textwidth]{img/CNN_Pooling.png}
      Taken from FIDLE
    \end{center}
  \end{textblock}

  \begin{textblock}{45}(5, 75)
    Crude downsampling (dimensionality reduction)\\
    %\hyperlink{CNN_Overview}{\beamergotobutton{Go back}}
  \end{textblock}
\end{frame}


\begin{frame}
  \note{
    \begin{itemize}
    \item We can use back-propagation but we must do some computation
    \item Don't present the computation
    \end{itemize}
  }
  \frametitle{\acl{CNN}: Learning}

  \begin{textblock}{90}(5, 15)
    \begin{itemize}
    \item<1-> It's a feed-forward model so we know we can use back-propagation
    \item<2-> We have to compute the gradients for convolutional layers!
    \end{itemize}
  \end{textblock}

  \begin{textblock}{90}(5, 30)
    \onslide<3->{
      \begin{center}
        \includegraphics[width=\textwidth]{img/Backpropagation_CNN.png}
      \end{center}
      Taken from FIDLE (notations slightly different)
    }
  \end{textblock}
\end{frame}


\begin{frame}
  \note{
    \begin{itemize}
    \item Present results:
      \begin{itemize}
      \item 3 times less errors => 3 times less manual intervention when sorting
        letters
      \item Results IRL
      \end{itemize}
    \end{itemize}
  }
  \frametitle{\acl{CNN}: Results}

  \begin{textblock}{90}(5, 15)
    \begin{block}{Results}
      \begin{itemize}
      \item Error rate on MNIST test set $< 1\%$
      \item<2-> Competitive with other methods
      \item<3-> Integrated in some cheque reading systems in 1996
      \item<4-> Much better than \ac{MLP} for other tasks/datasets
      \end{itemize}
    \end{block}
  \end{textblock}

\end{frame}


\begin{frame}
  \note{
    \begin{itemize}
    \item Activation maps are image-like
    \end{itemize}
  }
  \frametitle{\acl{CNN}: Interpretation}

  \begin{textblock}{90}(5, 15)
    \begin{block}{Interpretation}
      \onslide<2->{
        Plot the activations:
        \begin{center}
          \includegraphics[width=\textwidth]{img/CNN_Interpretation.png}
        \end{center}
        First layers are easier to interpret
      }
      \end{block}
  \end{textblock}

\end{frame}


\begin{frame}
  \frametitle{What have we learned so far?}

  \begin{textblock}{90}(5, 15)
    \begin{itemize}
    \item<1-> No magic!
      \begin{itemize}
      \item YOU have to do the math
      \end{itemize}
    \item<2-> A bit of Neural Networks:
      \begin{itemize}
      \item Neurons: weighted sum and activation function
      \item Architecture: layers and weights
      \item Learning: back-propagation
      \end{itemize}
    \item<3-> Neural networks are very flexible:
      \begin{itemize}
      \item Mathematical foundations are not great, not terrible
      \item Be very careful with biological analogies
      \end{itemize}
    \item<4-> ML is not only about models:
      \begin{itemize}
      \item Understand the data to design a model
      \item Need a lot of data \& computational power
      \item Data hell
      \item It's a job!
      \end{itemize}
    \end{itemize}
  \end{textblock}
\end{frame}


\appendix

% Appendices
% To be used with hyper links

\begin{frame}[label=Backpropagation]
  \frametitle{\acl{MLP}: Compute the gradient (1/2)}

  \begin{textblock}{90}(5, 15)
    \begin{itemize}
    \item Gradient:
      $
      \grad_{\params} \apply{\lossFunc}{\apply{\hyp_{\params}}{\example_i}, \knownLabel_i} =
      \left( \partialDeriv{\apply{\lossFunc}{\apply{\hyp_{\params}}{\example_i},
            \knownLabel_i}}{w_{1,1}^1}, \ldots, \partialDeriv{\apply{\lossFunc}{\apply{\hyp_{\params}}{\example_i}, \knownLabel_i}}{w_{k,p}^l},\ldots,\partialDeriv{\apply{\lossFunc}{\apply{\hyp_{\params}}{\example_i}, \knownLabel_i}}{w_{N_{4},N_{3}}^{3}} \right)^T
      $

      % $
      % \grad_{\params} \partialDeriv{1}{m} \sum_{i=1}^m \apply{\lossFunc}{\apply{\hyp_{\params}}{\example_i}, \knownLabel_i} =
      % \partialDeriv{1}{m} \sum_{i=1}^m \left( \partialDeriv{\apply{\lossFunc}{\apply{\hyp_{\params}}{\example_i}, \knownLabel_i} }{w_{1,1}^1},\ldots,\partialDeriv{\apply{\lossFunc}{\apply{\hyp_{\params}}{\example_i}, \knownLabel_i} }{w_{k,p}^l},\ldots,\partialDeriv{\apply{\lossFunc}{\apply{\hyp_{\params}}{\example_i}, \knownLabel_i} }{w_{N_{L-1},N_{L}}^{N_{L-1}}} \right)^T
      % $
      \begin{itemize}
      \item Analytical calculation is possible but tedious
      \end{itemize}
    \item $\hyp_{\params}$ is a composition of functions:
      \[
        \apply{\hyp_{\params}}{\example} = \apply{\softMaxFunc}{%
          \weightMatrix^3 \times
          \apply{\ReLUFunc}{%
            \weightMatrix^2 \times
            \apply{\ReLUFunc}{%
              \weightMatrix^1 \times \example
            }
          }
        }
      \]
      \begin{itemize}
      \item Use the chain rule to compute partial derivatives.
      \end{itemize}
      % $
      % \partialDeriv{\apply{\lossFunc}{\apply{\hyp_{\params}}{\example_i}, \knownLabel_i}}{w_{k,p}^l} =
      % \partialDeriv{\apply{\lossFunc}{\apply{\hyp_{\params}}{\example_i}, \knownLabel_i}}{o_k^{l+1}} \partialDeriv{o_k^{l+1}}{s_k^{l+1}} \partialDeriv{s_k^{l+1}}{w_{k,p}^{l}}
      % $
    \end{itemize}
  \end{textblock}
\end{frame}

\begin{frame}
  \frametitle{\acl{MLP}: Compute the gradient (2/2)}

  \begin{textblock}{90}(5, 15)
    \begin{block}{Derivation of the gradient}
      \begin{itemize}
      \item Apply the chain rule at layer $\ell$:
        $
        \partialDeriv{\apply{\lossFunc}{\apply{\hyp_{\params}}{\example_i}, \knownLabel_i}}{w_{k,p}^\ell} =
        \partialDeriv{\apply{\lossFunc}{\apply{\hyp_{\params}}{\example_i}, \knownLabel_i}}{o_k^{\ell+1}} \partialDeriv{o_k^{\ell+1}}{s_k^{\ell+1}} \partialDeriv{s_k^{\ell+1}}{w_{k,p}^{\ell}}
        $

      \item $\partialDeriv{\apply{\lossFunc}{\apply{\hyp_{\params}}{\example_i}, \knownLabel_i}}{o_k^{\ell+1}}$: known at the $\ell+1$ layer. $\partialDeriv{\apply{\lossFunc}{\apply{\hyp_{\params}}{\example_i}, \knownLabel_i}}{o_k^{L}}$: last layer, $\lossFunc$ must be differentiable w.r.t $o_k^{L}$.

    %\item $\partialDeriv{\apply{\lossFunc}{\apply{\hyp_{\params}}{\example_i}, \knownLabel_i}}{o_k^{L}}$: last layer, $\lossFunc$ must be differentiable w.r.t $o_k^{L}$

      \item $\partialDeriv{o_k^{\ell+1}}{s_k^{\ell+1}}$: $o_k^{\ell+1} = \activFunc(s_k^{\ell+1})$, $\activFunc$ as intermediate or final activation function must be differentiable.

      \item $\partialDeriv{s_k^{\ell+1}}{w_{k,p}^{\ell}} = \partialDeriv{\partial}{w_{k,p}^{\ell}} \sum_{p' \in \left\{1,\ldots,N_l \right\}}  w_{p', k}^{\ell}o_{p'}^{k}= \partialDeriv{}{w_{k,p}^{\ell}}  w_{p, k}^{\ell}o_{p}^{\ell}=o_{p}^{\ell}$

    \end{itemize}
    \end{block}
  \end{textblock}

  \begin{textblock}{90}(5, 65)
    \begin{block}{General idea}<2->
      \begin{itemize}
      \item Back-propagation works for all feed-forward architectures
        \hyperlink{MLP_Learning_2}{\beamergotobutton{Go back}}
        % Back-propagates to main slide :)
      \end{itemize}
    \end{block}
  \end{textblock}
\end{frame}


% Appendices
% To be used with hyper links

\begin{frame}[label=Convolutional_Layers]
  \frametitle{\acl{CNN}: Convolutional Layers}

  \begin{textblock}{90}(5, 10)
    \begin{center}
      \includegraphics[width=\textwidth]{img/Convolution.png}
      Taken from FIDLE
    \end{center}
  \end{textblock}

  % Cheat a bit on the size
  \begin{textblock}{46}(5, 75)
    \begin{itemize}
    \item Common operation in image processing
    \item $\matrix{K}$ acts as a filter
    \end{itemize}
    \hyperlink{CNN_Overview_Convolutional}{\beamergotobutton{Go back}}
  \end{textblock}

  \begin{textblock}{45}(50, 75)
    \includegraphics[width=\textwidth]{img/CNN_Filter.png}
  \end{textblock}
\end{frame}


\begin{frame}[label=Pooling_Layers]
  \frametitle{\acl{CNN}: Pooling Layers}

  \begin{textblock}{90}(5, 10)
    \begin{center}
      \includegraphics[width=\textwidth]{img/CNN_Pooling.png}
      Taken from FIDLE
    \end{center}
  \end{textblock}

  \begin{textblock}{90}(5, 75)
    Crude downsampling (dimensionality reduction)\\
    \hyperlink{CNN_Overview_Pooling}{\beamergotobutton{Go back}}
  \end{textblock}
\end{frame}


\end{document}
