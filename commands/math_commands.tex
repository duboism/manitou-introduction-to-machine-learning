%
% Math commands
%
% $Author: duboism $
% $LastChangedDate: 2013-01-15 16:56:30 +0100 (mar. 15 janv. 2013) $

%\usepackage{amsmath, amssymb, amsthm, textcomp, stmaryrd}
%\usepackage{textcase}

%%%%%%%%%%%
% General %
%%%%%%%%%%%

%\newtheorem{definition}{Définition}
%\newtheorem{thm}{Théorème}
%\newtheorem{prop}{Proposition}
%\newtheorem{exemple}{Exemple}
%\newtheorem{NB}{NB}

% Sets
\def\setextleftdelim{\lbrace}
\def\setextrightdelim{\rbrace}
\newcommand{\setext}[1]{\ensuremath{\left \setextleftdelim {#1} \right \setextrightdelim}} % Ensemble en extension: {x1, ..., xn}
\def\setpredicatesep{,}
\newcommand{\setpredicate}[2]{\ensuremath{\left \setextleftdelim {#1} \setpredicatesep {#2} \right \setextrightdelim}} % Ensemble en intension: {xi, 1<= i <= n}
\def\setstyle{\mathcal}
\newcommand{\set}[1]{\ensuremath{\setstyle{#1}}}
\newcommand{\card}[1]{\ensuremath{\left | #1 \right |}}

% Well known sets
\def\setN{\mathbb{N}}
\def\setZ{\mathbb{Z}}
\def\setQ{\mathbb{Q}}
\def\setR{\mathbb{R}}
\def\setC{\mathbb{C}}

% Tuples
\def\tupleleftdelim{(}
\def\tuplerightdelim{)}
\newcommand{\couple}[2]{\ensuremath{\left \tupleleftdelim {#1}, {#2} \right \tuplerightdelim}}
\newcommand{\tupleext}[1]{\ensuremath{\left \tupleleftdelim {#1} \right \tuplerightdelim}} % Tuple en extension: (x1, ..., xn)
\def\tupleintsep{_}
\newcommand{\tupleint}[2]{\ensuremath{\left \tupleleftdelim {#1} \right \tuplerightdelim \tupleintsep {#2}}} % Tuple en intension: (xi)_{1 <= i <= n}
\def\tuplestyle{\mathbf}
\newcommand{\tuple}[1]{\ensuremath{\tuplestyle{#1}}}


% Intervals: http://en.wikipedia.org/wiki/Interval_%28mathematics%29#Notations_for_intervals
\def\intervalleftdelim{[}
\def\intervalrightdelim{]}
\def\intervallodelim{\intervalrightdelim} % left-open symbol
\def\intervalrodelim{\intervalleftdelim} % right-open symbol
\newcommand{\interval}[2]{\ensuremath{\left \intervalleftdelim {#1}, {#2} \right \intervalrightdelim}}
\newcommand{\lointerval}[2]{\ensuremath{\left \intervallodelim {#1}, {#2} \right \intervalrightdelim}}
\newcommand{\rointerval}[2]{\ensuremath{\left \intervalleftdelim {#1}, {#2} \right \intervalrodelim}}
\newcommand{\ointerval}[2]{\ensuremath{\left \intervallodelim {#1}, {#2} \right \intervalrodelim}}

% Integer intervals
\def\intintervalleftdelim{\llbracket}
\def\intintervalrightdelim{\rrbracket}
\def\intintervallodelim{\intintervalrightdelim} % left-open symbol
\def\intintervalrodelim{\intintervalleftdelim} % right-open symbol
\newcommand{\intinterval}[2]{\ensuremath{\left \intintervalleftdelim {#1}, {#2} \right \intintervalrightdelim}}
\newcommand{\lointinterval}[2]{\ensuremath{\left \intintervallodelim {#1}, {#2} \right \intintervalrightdelim}}
\newcommand{\rointinterval}[2]{\ensuremath{\left \intintervalleftdelim {#1}, {#2} \right \intintervalrodelim}}
\newcommand{\ointinterval}[2]{\ensuremath{\left \intintervallodelim {#1}, {#2} \right \intintervalrodelim}}

% Functions and operators
\def\functionstyle{\mathrm}
\newcommand{\func}[1]{\ensuremath{\functionstyle{#1}}}
\newcommand{\apply}[2]{\ensuremath{#1 \left ( {#2} \right )}}
\def\opstyle{\mathrm}
\newcommand{\op}[1]{\ensuremath{\opstyle{#1}}}
\newcommand{\applyop}[2]{\ensuremath{#1 \left [ {#2} \right ]}}

% Vectors, Matrices, ...
\def\vectorextleftdelim{[}
\def\vectorextrightdelim{]}
\newcommand{\vectorext}[1]{\ensuremath{\left \vectorextleftdelim {#1} \right \vectorextrightdelim}}
\newcommand{\vectorstyle}[1]{\mathbf{#1}}
\renewcommand{\vector}[1]{\ensuremath{\vectorstyle{#1}}}
\newcommand{\vectorElem}[2]{\ensuremath{#1 \left [ #2 \right ]}}

\def\matrixextleftdelim{[}
\def\matrixextrightdelim{]}
\newcommand{\matrixext}[1]{\ensuremath{\left \matrixextleftdelim {#1} \right \matrixrightdelim}}
\def\matrixstyle{\mathbf}
\renewcommand{\matrix}[1]{\ensuremath{\matrixstyle{#1}}}
\newcommand{\matrixElem}[3]{\ensuremath{#1 \left [ {#2}, {#3} \right ]}}
\newcommand{\matrixLinElem}[2]{\ensuremath{#1 \left [ #2 \right ]}}

% Scalar product
\def\dotprodleftdelim{\left \langle}
\def\dotprodrightdelim{\right \rangle}
\def\dotprodcentraldelim{,}
\newcommand{\dotprod}[2]{\ensuremath{\dotprodleftdelim #1 \dotprodcentraldelim #2 \dotprodrightdelim}}

% Vector product
\def\vectorprodsymbol{\times}
\newcommand{\vectorprod}[2]{#1 \vectorprodsymbol #2}

% TODO:
% - sets: set builder notation? http://en.wikipedia.org/wiki/Set_builder_notation
% - set theory: define synonyms for common opertaions (union <-> \bigcup, intersection <-> \bigcap, etc.)
% - matrix
% - indexing should be done on class basis: setelement, matrix element, imageelement, ...
% - limits: use \def instead of \newcommand?
% - probabilities: use \DeclareMathOperator? There is a \Pr operator in \LaTeX

% Integral
\def\dsymb{\mathrm{d}}
\newcommand{\intover}[1]{\dsymb #1}

% Limits
\newcommand{\tendsto}[1]{\ensuremath{\rightarrow #1}}
\newcommand{\tendstoP}{\ensuremath{\overset{P}{\rightarrow}}}

% Optimal quantity
\def\optsymbol{\star}
\newcommand{\opt}[1]{\ensuremath{#1^{\optsymbol}}}

% Estimation of a quantity
\def\estimsymbol{\widehat}
\newcommand{\estim}[1]{\ensuremath{\estimsymbol{#1}}}

%%%%%%%%%%%%%%%%%
% Probabilities %
%%%%%%%%%%%%%%%%%
\newcommand{\RV}[1]{\ensuremath{\MakeTextUppercase{#1}}}
\newcommand{\cond}[2]{{#1} | {#2}}

\def\drawFromSymbol{\sim}
\newcommand{\drawFrom}[2]{\ensuremath{#1 \drawFromSymbol #2}}

% Discrete probability
\def\probSymbol{P}
\newcommand{\prob}[1]{\ensuremath{\apply{\func{\probSymbol}}{#1}}}
\newcommand{\condprob}[2]{\ensuremath{\prob{\cond{#1}{#2}}}}

% PDF
\def\pdfSymbol{p}
\newcommand{\pdf}[1]{\ensuremath{\apply{\func{\pdfSymbol}}{#1}}}
\newcommand{\condpdf}[2]{\ensuremath{\pdf{\cond{#1}{#2}}}}

% Operators
\newcommand{\expectation}[1]{\ensuremath{\applyop{\op{\mathbb{E}}}{#1}}}
\newcommand{\var}[1]{\ensuremath{\applyop{\op{Var}}{#1}}}


%%%%%%%%%%%%%%%%%%%%%%
% General operations %
%%%%%%%%%%%%%%%%%%%%%%

% Argmax & friends
\DeclareMathOperator*{\argmax}{arg\,max}
\DeclareMathOperator*{\argmin}{arg\,min}

% BigO and friends
% From: http://www.tug.org/pipermail/texhax/2004-January/001496.html
% See also: http://xw2k.nist.gov/dads//HTML/bigOnotation.html
\DeclareMathOperator{\BigOmicron}{O}
\DeclareMathOperator{\littleomicron}{o}
\DeclareMathOperator{\BigOmega}{\Omega}
\DeclareMathOperator{\littleomega}{\omega}
\DeclareMathOperator{\BigTheta}{\Theta}

\newcommand{\boundedabove}[1]{\BigOmicron \left ( {#1} \right )}
\newcommand{\dominated}[1]{\littleomicron \left ( {#1} \right )}

\newcommand{\boundedbelow}[1]{\BigOmega \left ( {#1} \right )}
\newcommand{\dominates}[1]{\littleomega \left ( {#1} \right )}

\newcommand{\boundedabovebelow}[1]{\BigTheta \left ( {#1} \right )}
\newcommand{\similar}[2]{{#1} \sim {#2}}

% Norms
\def\normleftdelim{\|}
\def\normrightdelim{\|}
\newcommand{\pnorm}[2]{\ensuremath{\left \normleftdelim {#1} \right \normrightdelim_{#2}}}
\newcommand{\norm}[1]{\ensuremath{\left \normleftdelim {#1} \right \normrightdelim}}
\newcommand{\abs}[1]{\ensuremath{\left | {#1} \right |}}

% Special functions
\newcommand{\ceil}[1]{\ensuremath{\left \lceil {#1} \right \rceil}}
\newcommand{\floor}[1]{\ensuremath{\left \lfloor {#1} \right \rfloor}}
\DeclareMathOperator{\sgn}{sgn}

% Convolution
\def\convsymbol{\star}
\DeclareMathOperator*{\conv}{\convsymbol}
\newcommand{\convvar}[1]{\ensuremath{\conv_{#1}}} % argument is used to specify convolution variables

% Derivatives
\newcommand{\partialDeriv}[2]{\frac{\partial #1}{\partial #2}}

%%%%%%%%%%%%%%%%%%%%
% Image processing %
%%%%%%%%%%%%%%%%%%%%

\let\imagestyle\mathbf
\newcommand{\image}[1]{\ensuremath{\imagestyle{#1}}}
\newcommand{\imagepixel}[3]{\ensuremath{\image{#1} \left ( #2, #3 \right )}}
\newcommand{\gausskernel}[3]{\ensuremath{\frac{1}{\sqrt{2\pi} #3} \exp^{-\frac{(#1+#2)^2}{2 #3^2}} }}

% SIFT
\newcommand{\siftkernel}{\gausskernel{x}{y}{\sqrt{2}}}

%\newcommand{\convdd}[4]{\int_3 \int_4 #1 #2}

%%%%%%%%%%%%
% Learning %
%%%%%%%%%%%%

\DeclareMathOperator{\error}{error}

% Probabilistic robotics
\newcommand{\stateSymbol}{x}
\newcommand{\state}{\vector{\stateSymbol}}
\newcommand{\obsSymbol}{z}
\newcommand{\obs}{\vector{\obsSymbol}}
\newcommand{\controlSymbol}{u}
\newcommand{\control}{\vector{\controlSymbol}}
\newcommand{\aggState}{\tilde{x}}
\newcommand{\stateRVSymbol}{\RV{\stateSymbol}}
\newcommand{\stateRV}{\vector{\stateRVSymbol}}
\newcommand{\obsRVSymbol}{\RV{\obsSymbol}}
\newcommand{\obsRV}{\vector{\obsRVSymbol}}
\newcommand{\controlRVSymbol}{\RV{\controlSymbol}}
\newcommand{\controlRV}{\vector{\controlRVSymbol}}
\newcommand{\aggStateRV}{\RV{\aggState}}

\newcommand{\val}[2]{\ensuremath{{#1}_{#2}}}
\newcommand{\seq}[3]{\ensuremath{{#1}_{{#2}:{#3}}}}

\newcommand{\bel}{\func{bel}}
\newcommand{\belBeforeObs}{\overline{{\bel}}}

